% LTeX: language=fr-FR

Pour obtenir les meilleures performances possibles dans les supercalculateurs,
il est aujourd'hui nécessaire d'utiliser des réseaux d'interconnexion de plus en
plus sophistiqués. C'est pour répondre à ce besoin qu'Atos produit le réseau
BXI, composé de cartes réseau (NIC) et de commutateurs (switches). Le
paramétrage optimal de ce matériel est une problématique complexe, en
particulier parce que l'espace de paramètres que l'on souhaite explorer est
grand, et qu'il n'est pas possible de faire des expériences réelles permettant
d'explorer cet espace. Ainsi, on souhaite utiliser un simulateur pour évaluer
les performances d'une application donnée pour un ensemble de paramètres. 

La principale difficulté rencontrée lors du développement de tels simulateurs
provient de la complexité de la pile logicielle utilisée sur les grappes de
calcul que l'on souhaite simuler~: les applications scientifiques sont souvent
programmées à un haut niveau d'abstraction et leurs communications traversent
plusieurs couches logicielles avant d'être exécutées par le matériel. Il faut
donc choisir laquelle de ces couches logicielle intercepter en simulation, afin
de garantir une bonne précision tout en conservant des performances acceptables. 

Notre contribution prend la forme d'un simulateur du réseau BXI, qui propose un
modèle simplifié du matériel, tout en permettant l'exécution d'applications
complètes sur plusieurs machines simulées. Ce simulateur permet de prendre en
compte les spécificités de toutes les couches logicielles intermédiaires entre
le NIC et l'application simulée, moyennant des modifications mineures dans
celles-ci. Nous validons ce modèle expérimentalement en comparant l'exécution de
benchmarks sur une grappe de calcul équipée de matériel BXI et la simulation de
ces benchmarks dans notre simulateur. Ce modèle bas niveau ayant un coût en
performance non négligeable, nous présentons également une méthodologie pour
alterner dynamiquement entre plusieurs modèles de précision différente au cours
de l'exécution d'une simulation, afin de permettre à l'utilisateur de paramétrer
au mieux quelles parties de l'application doivent être simulées avec le plus de
précision.


\section*{Chapitre 1}

Ce chapitre introduit le contexte de nos travaux (décrit ci-dessus). Nous y
présentons quelques informations sur le processus de développement qu'utilisent
les équipes d'Atos pour créer le matériel BXI, qui sert de cas d'étude à nos
travaux. Ce premier chapitre termine par une annonce nos contribution et du plan
du document.

\section*{Chapitre 2}

Le deuxième chapitre précise le contexte de nos travaux en présentant le domaine
du calcul haute performance. Y sont présentées l'architecture d'une grappe de
calcul (sujet d'étude de nos expériences) ainsi que les caractéristiques du
matériel réseau développé par Atos, que nous cherchons à modéliser. En
particulier, nous présentons l'API Portals, qui est implémentée de manière
matérielle par l'interconnect BXI d'Atos. Le chapitre se termine par une
présentation des programmes informatiques que nous tâcherons de simuler par la
suite, afin d'expliquer leur intérêt dans le cadre de la validation de notre
simulateur.

\section*{Chapitre 3}

Le troisième chapitre présente l'état de l'art existant dans le domaine de la
simulation de systèmes HPC. Nous y présentons la principale difficulté auxquels
tous les simulateurs font face~: le compromis entre précision du modèle et
performance de la simulation. Nous détaillons plus amplement les
caractéristiques de quelques simulateurs qui se prêtent au type de travaux que
nous entreprenons, en particulier SimGrid que nous avons choisi d'utiliser.


\section*{Chapitre 4}

Le quatrième chapitre présente notre première contribution~: un simulateur de
l'API Portals, optimisé pour modéliser l'interconnect BXI d'Atos. Nous
détaillons en particulier notre description de platforme, qui modélise les
différents composants matériels présents dans une grappe de calcul réelle, ansi
que notre implémentation de l'API Portals dans le monde simulé, en utilisant les
briques de base offertes par SimGrid.

Nous présentons les outils qui ont été développés pour assiter l'utilisateur du
simulateur, et nous terminons avec des expériences de validation, qui comparent
nos résultats de simulations avec des exécutions de benchmarks sur une grappe de
calcul réelle (équipée de matériel BXI).

\section*{Chapitre 5}

Le cinquième chapitre explique comment des bibliothèques logicielles abstraites
peuvent être adaptées pour fonctionner au dessus de notre modèle bas niveau.
Nous étudions en particulier l'API MPI, car c'est la plus couramment utilisée
dans le domaine du calcul haute performance. Nous montrons que notre approche
permet d'exécuter MPI en simulation avec très peu de modifications, ce qui nous
permet de faire des simulations avec une pile logicielle beaucoup plus fidèle au
fonctionnement réel par rapport aux modèles fluides pré-existant dans l'état de
l'art.

Nous présentons une validation expérimentales, utilisant les programmes décrit
dans le Chapitre~2. Nous comparons nos résultat à SMPI, simulateur pré-existant
le plus proche de notre modèle. Nous en concluons que les deux modèles ont des
forces et faiblesses complémentaires~: là où notre approche (basée sur notre
simulateur, S4BXI) offre un modèle plus précis, SMPI offre un modèle plus rapide
d'exécution.

Après avoir étudié MPI, nous présentons des travaux préliminaires sur la
simulation de l'API OpenSHMEM. Nous détaillons comment adapter cette
bibliothèque à notre simulateur, et présentons des résultats expérimentaux.

\section*{Chapitre 6}

Ce chapitre présente une nouvelle contribution~: en se basant sur les forces et
les faiblesses des deux simulateurs que nous avons étudiés, S4BXI et SMPI, nous
développons une méthodologie de simulation permettant de coupler les deux
modèles au sein d'une même simulation. Ainsi, les utilisateurs de notre
simulateurs peuvent changer de modèle réseau dynamiquement, durant l'exécution
de la simulation.

Nous présentons des résultats expérimentaux utilisant cette approche, sur deux
programmes simulés. Nous proposons une conclusion qui met en avant les forces et
les faiblesses de cette méthodologie de simulation.

\section*{Chapitre 7}

Nous dédions un chapitre complet à la présentation des études futures qui
pourraient être réalisées à partir de nos travaux. Nous présentons des exemples
d'expériences sur lesquelles nous avons commencé à travailler, mais qui
nécessiteraient plus de temps de travail. Nous détaillons en particulier une
étude s'intéressant à la notion de contrôle de flux, dans le but de simplifer le
processus de développement des prochaines générations de matériel BXI.

\section*{Chapitre 8}

Dans ce dernier chapitre, nous présentons la conclusion de nos travaux~: nous
résumons nos contributions principales, ainsi que les portes que nos travaux
ouvrent pour des études futures.