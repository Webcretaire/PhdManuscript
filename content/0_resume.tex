% LTeX: language=fr-FR

\chapter*{Résumé}
\addcontentsline{toc}{chapter}{Résumé} 

Pour obtenir les meilleures performances possibles dans les supercalculateurs,
il est aujourd'hui nécessaire d'utiliser des réseaux d'interconnexion de plus en
plus sophistiqués. C'est pour répondre à ce besoin qu'Atos produit le réseau
BXI, composé de cartes réseau (NIC) et de commutateurs (switches). Le
paramétrage optimal de ce matériel est une problématique complexe, en
particulier parce que l'espace de paramètres que l'on souhaite explorer est
grand, et qu'il n'est pas possible de faire des expériences réelles permettant
d'explorer cet espace. Ainsi, on souhaite utiliser un simulateur pour évaluer
les performances d'une application donnée pour un ensemble de paramètres. 

La principale difficulté rencontrée lors du développement de tels simulateurs
provient de la complexité de la pile logicielle utilisée sur les grappes de
calcul que l'on souhaite simuler~: les applications scientifiques sont souvent
programmées à un haut niveau d'abstraction et leurs communications traversent
plusieurs couches logicielles avant d'être exécutées par le matériel. Il faut
donc choisir laquelle de ces couches logicielle intercepter en simulation, afin
de garantir une bonne précision tout en conservant des performances acceptables. 

Notre contribution prend la forme d'un simulateur du réseau BXI, qui propose un
modèle simplifié du matériel, tout en permettant l'exécution d'applications
complètes sur plusieurs machines simulées. Ce simulateur permet de prendre en
compte les spécificités de toutes les couches logicielles intermédiaires entre
le NIC et l'application simulée, moyennant des modifications mineures dans
celles-ci. Nous validons ce modèle expérimentalement en comparant l'exécution de
benchmarks sur une grappe de calcul équipée de matériel BXI et la simulation de
ces benchmarks dans notre simulateur. Ce modèle bas niveau ayant un coût en
performance non négligeable, nous présentons également une méthodologie pour
alterner dynamiquement entre plusieurs modèles de précision différente au cours
de l'exécution d'une simulation, afin de permettre à l'utilisateur de paramétrer
au mieux quelles parties de l'application doivent être simulées avec le plus de
précision.

% LTeX: language=en-US

\makeatletter
\@openrightfalse

\chapter*{Abstract}
\addcontentsline{toc}{chapter}{Abstract} 

In order to obtain the best possible performance in supercomputers, it is
necessary to use interconnection networks of ever-increasing complexity. In
order to provide such networks, Atos designs the BXI interconnect, which is
composed of network controllers (NIC) and switches. Optimal configuration of
this hardware is a complex task, in particular because the parameter space to
explore is very large, and it is not practically possible to run real-world
experiments to explore this parameter space. For this reason, we wish to use a
simulator to evaluate the performance of a given application for a set of
parameters.

The main difficulty that arises when developing such simulators comes from the
complexity of the software stack that is used on the clusters that we wish to
model: scientific applications are often programmed at a high level of
abstraction, and their communications go through several layers of software
before they are executed by the hardware. Therefore, it is important to choose
which software layer to intercept in simulation, in order to get a good accuracy
of the model while keeping acceptable performance of the simulation.

Our contribution is a simulator of the BXI interconnect, which provides a
simplified model of the hardware, while allowing the execution of complete
applications of several simulated machines. This simulator accounts for the
properties of all layers of the software stack between the NIC and the simulated
application, thanks to minor modifications in these software libraries. We
validate this model experimentally by comparing the execution of benchmarks on a
cluster equipped with BXI hardware and the simulation of these benchmarks in our
simulator. Because our low-level model has a significant cost in terms of
performance, we also present a methodology to switch dynamically between several
models of different accuracy during the execution of a simulation, in order to
allow users to tune as well as possible which parts of the application should be
modeled with the best accuracy.

\@openrighttrue
\makeatother
